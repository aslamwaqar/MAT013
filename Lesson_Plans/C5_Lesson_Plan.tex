\documentclass[12pt]{article}

\setlength{\parindent}{0pt}
\setlength{\parskip}{.5cm}
\setlength{\textheight}{220mm}
\setlength{\textwidth}{166mm}
\setlength{\topmargin}{0mm}
\setlength{\oddsidemargin}{0mm}

\usepackage{enumitem}
\pagestyle{empty}
\begin{document}

\begin{center}
\huge{MAT013: C5 - Lesson plan}
\end{center}
\begin{center}
\tiny{(This sheet was last updated on \today)}
\end{center}



\section{Description}

This is a lesson plan for morning long class delivering content relevant to chapter 5 of MAT013. Which will cover:

\begin{itemize}
\item Three SAS tools: sql, functions and optimisation.
\item Three R packages: sqldf, ggplot2 and twitteR.
\end{itemize}

The intended learning outcomes for this day.

On completion of the class a student should be able to:

\begin{enumerate}[label=\Alph*]
\item Use sql within SAS and R.
\item Build functions in SAS.
\item Solve various optimisation problems in SAS.
\item Construct publication quality graphics in R.
\item Data mine twitter with R.
\end{enumerate}


\section{Lesson plan}
\begin{center}
\begin{tabular}{|p{2cm}|p{2cm}|p{4.5cm}|p{4.5cm}|p{2cm}|}
\hline
Time&ILO&Teacher Activity& Learner Activity& Resources\\\hline
Before the class&A,B,C,D,E&Invite student to look at videos and distribute challenges&View videos and as a group carry out challenges.&Videos, Challenges and Notes.\\\hline
0-20mins&A,B,C,D,E&Listen&Listen and deliver group SAS activity&Student presentations\\\hline
20-40mins&A,B,C,D,E&Discuss&Discuss&Lecturer slides + Student presentation\\\hline
40-100mins&A,B,C,D&Assist&Carry out Lab sheet, Exercise sheet or work on challenge (student choice) &SAS Lab Sheets\\\hline
100-120mins&NA&NA&Break&Coffee\\\hline
120-140mins&A,B,C,D,E&Listen&Listen and deliver group R activity&Student presentations\\\hline
140-180mins&A,B,C,D,E&Discuss&Discuss&Lecturer slides + Student presentation\\\hline
180-240mins&A,B,C,D&Assist&Carry out Lab sheet, Exercise sheet or work on challenge (student choice) &R Lab Sheets\\\hline
\end{tabular}
\end{center}

\section{Assessment}
The above ILOs will be assessed through the various assessments at the end of the module. ILOs A, B and C will be evaluated as a consequence of the students being able to use SAS and R.


\end{document}
