\documentclass[12pt]{article}

%packages
\usepackage{enumerate}
\usepackage{graphicx}
\usepackage{amsmath}
\usepackage{mathdots}
\usepackage{amsthm}
\usepackage{amssymb}
\usepackage{hyperref}
%Margins etc...
\setlength{\textheight}{240mm}
\setlength{\topmargin}{-17mm} \setlength{\oddsidemargin}{-4mm}
\setlength{\textwidth}{166mm} \setlength{\parindent}{0mm}
\setlength{\marginparsep}{9mm} \setlength{\parskip}{3mm}
\pagestyle{empty}

\begin{document}
\begin{center}
\huge{MAT013 - SAS: Chapter 4 Lab Sheet}\\
\begin{center}
\tiny{(This sheet was last updated on \today)}
\end{center}
\end{center}

\begin{enumerate}
\item Create a (single) data set containing the name of the observations from ``JJJ'' and ``MMM'' as well as a new variable which is ``Y'' if the individual is clinically obese and ``N'' otherwise.
\item Create a (single) data set containing the total number of birthday candles used throughout the lives of every individual from both ``JJJ'' and ``MMM''.
\item Obtain the first even numbers less than 240.
\item Create a macro that outputs a scatter plot of height against weight for observations in the ``JJJ''  and ``MMM'' data sets? Modify the macro so that it outputs the plot to a pdf file.
\item Create a macro that computes the left over life savings after a given quantity of spending on a given quantiy of shopping trips from the ``JJJ'' data set.
\item Modify the above macro so that a default value is given to spend of 430 and a default value of 3 trips.
\item Use the `\%let' statement to pass a value to the above macro.
\item Use the `\%put' statement to show all the local and global variables.
\item Modify the above macro so that two different data sets are created depending on whether or not spend is positive or negative. Output a message to the log if the spend is 0.
\item Create a macro that creates 15 data sets each with updated savings in pounds for observations in the ``JJJ'' and ``MMM'' data sets for varying values for the number of trips (1 to 15).
\item Download the files \href{https://docs.google.com/file/d/0Bx_zrw5uAafbc3B1WndVdk1FM1E/edit}{``File\_1.csv - File\_200.csv''} and create a function that automatically imports them.
\end{enumerate}




\end{document}
